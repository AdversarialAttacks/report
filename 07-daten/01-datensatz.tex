\section{Daten}

In unserer Arbeit konzentrieren wir uns primär auf zwei Datensätze, die jeweils medizinische Bilder von Patienten enthalten, die entweder an COVID-19 erkrankt sind (im COVIDx CXR-4 Datensatz) oder an Gehirntumoren leiden (im Brain Tumor Datensatz). Diese Datensätze weisen die Gemeinsamkeit auf, dass sie medizinische Bildinformationen enthalten. 

\subsection{COVIDx CXR-4}
\textbf{COVIDx CXR-4} \cite{wu_covidx_2023} ist ein öffentlicher Datensatz für COVID-19-Diagnostik mit Röntgenbildern, der 84,818 Bilder von 45,342 Patienten enthält. COVIDx CXR-4 ist, nach Kenntnisstand der Autoren, der grösste und vielfältigste öffentlich verfügbare COVID-19-Datensatz für Röntgenbilder und soll die Forschung unterstützen, um Klinikern im Kampf gegen COVID-19 zu helfen.


\subsubsection{Datenpartitionierung}

Die Datenpartitionierung ist bereits durch die Struktur vorgegeben. Wir stellen fest, dass die Klassenverteilung von positiven und negativen Labels für die Validierung und den Testdatenset ungefähr gleichverteilt ist, mit einem Verhältnis von 50\%. 

\begin{table}[h]
\centering
\begin{tabular}{@{}cccccc@{}}
\toprule
Partition & \multicolumn{2}{c}{Anzahl Bilder} & \multicolumn{2}{c}{Klassenverteilung} & Positiv-Verhältnis\\ 
\cmidrule(lr){2-3} \cmidrule(lr){4-5} 
          & Absolut & Relativ & Positiv & Negativ & \\ 
\midrule
Train      & 67863 & 0.8001 & 57199 & 10664 & 0.8429 \\
Validation & 8473  & 0.0999 & 4241  & 4232  & 0.5005 \\
Test       & 8482  & 0.1000 & 4241  & 4241  & 0.5000 \\ 
\bottomrule
\end{tabular}
\caption{Klassenverteilung von COVIDX-CXR4}
\label{tab:covidx-klassenverteilung}
\end{table}



\newpage
\subsection{Brain Tumor}
Der \textbf{Brain Tumor Classification (MRI)-Datensatz} \cite{bhuvaji_brain_2020} umfasst 3.260 bereinigte und augmentierte T1-gewichtete, kontrastverstärkte MRI-Bilder zur Identifikation und Klassifikation von Gehirntumoren.

\subsubsection{Datenpartitionierung}

Der Datensatz enthält vier Klassen, drei davon sind Tumorklassen und eine Klasse ist kein Tumor. Die Verteilung der Klassen ist in der Tabelle \ref{tab:mri-orginale-klassenverteilung} zu sehen. Ersichtlich ist, dass in dem Datensatz positive Tumorklassen deutlich mehr vertreten sind als keine Tumore. Da wir uns für unsere These auf die binäre Klassifikation stützen, fassen wir die drei Gehirntumoren Klassen Pituitary, Glioma, Meningioma zu einer positiven Gehirn Tumor Klasse und Kein Tumor als negative Klasse und erhalten somit die Tabelle \ref{tab:mri-binaere-klassenverteilung}.

\begin{table}[h]
\centering
\begin{tabular}{@{}ccccc@{}}
\toprule
 Partition & \multicolumn{4}{c}{Klassenverteilung}        \\ 
\cmidrule(l){2-5}
           & Pituitary & Glioma & Meningioma & Kein Tumor \\ 
\midrule 
Train      & 662 & 661 & 658 & 317 \\
Validation & 165 & 165 & 164 & 78  \\
Test       & 74  & 100 & 115 & 105 \\ 
\bottomrule
\end{tabular}
\caption{Ursprüngliche Klassenaufteilung von Gehirntumoren}
\label{tab:mri-orginale-klassenverteilung}
\end{table}

Die Verteilungsverhältnisse waren ursprünglich 70.4\% Trainings- und 29.6\% Testbilder. Da ein analoger Datensatz, wie beim COVIDX, einfacher zu handhaben ist, haben wir den ursprünglichen Testdatensatz weiter unterteilt, und zwar in 17.5\% Validierungs- und 12.1\% Testdaten. Bevor wir die positiven Klassen zusammengefasst haben, herrschte eine Klassenimbalance, die wir bei der Partitionierung in Train-, Validierung, und Testset mitberücksichtigt haben. Die Tabelle \ref{tab:mri-binaere-klassenverteilung} zeigt die Anzahl an Bilder in absolut, relativ und die Klassenverteilung von positive, negative Tumorbilder für jede Datenpartition auf.

\begin{table}[h]
\centering
\begin{tabular}{@{}cccccc@{}}
\toprule
Partition & \multicolumn{2}{c}{Anzahl Bilder} & \multicolumn{2}{c}{Klassenverteilung} & Positiv-Verhältnis\\ 
\cmidrule(lr){2-3} \cmidrule(lr){4-5} 
           & Absolut & Relativ & Positiv & Negativ \\ 
\midrule
Train      & 2298 & 0.704 & 1981 & 317 & 0.862 \\
Validation & 572  & 0.175 & 494  & 78  & 0.864 \\
Test       & 394  & 0.121 & 289  & 105 & 0.736 \\ 
\bottomrule
\end{tabular}
\caption{Binäre Klassenverteilung von Gehirntumoren}
\label{tab:mri-binaere-klassenverteilung}
\end{table}