\section{Diskussion und Ausblick}

\subsection{Auswirkungen unserer Ergebnisse}
Mögliche Gefahren, wichtige Punkte für die Forschung, etc.

\subsection{Opportunities}
\subsubsection{Bias Detection}
Mittels der Generation der Perturbationen könnte man auf Biases eines Modells schliessen, da die attakierte Stelle ja wichtig ist für das Modell. (Alternative zu Grad-Cam)

\subsection{Weitere mögliche Verteidigungsmechanismen}



\subsubsection{Data Augmentation}

\todo{zu wenig Zeit}

Bei der Data Augmentation verändern wir mit einer zufälligen Transformation die vorhandenen Trainingsdaten, um die Variabilität im Datensatz zu erhöhen. Dies kann durch verschiedene Methoden geschehen, wie zum Beispiel:

\begin{itemize}
    \item \textbf{Rotation}: Drehen des Bildes um einen bestimmten Winkel.
    \item \textbf{Verschiebung}: Verschieben des Bildes in eine bestimmte Richtung.
    \item \textbf{Skalierung}: Ändern der Grösse des Bildes.
    \item \textbf{Spiegelung}: Spiegeln des Bildes entlang einer Achse.
    \item \textbf{Rauschen hinzufügen}: Zufälliges Rauschen hinzufügen, um das Bild zu verändern.
\end{itemize}

Diese Techniken helfen, das Modell robuster zu machen und die Generalisierungsfähigkeit zu verbessern, indem sie es zwingen, verschiedene Variationen der Daten zu lernen.

\subsubsection{Input Ensembles}

\todo{zu wenig Zeit}

Mit Input Ensembles nutzen wir mehrere Modelle, die unabhängig voneinander trainiert wurden. Der Prozess sieht wie folgt aus:

\begin{enumerate}
    \item \textbf{Training mehrerer Modelle}: Jedes Modell wird separat mit demselben Trainingsdatensatz trainiert.
    \item \textbf{Unabhängige Vorhersagen}: Jedes Modell gibt eine eigene Vorhersage, basierend auf dem Input.
    \item \textbf{Kombinieren der Vorhersagen}: Der endgültige Output wird durch Mehrheitsentscheidung (Voting) bestimmt. 
\end{enumerate}

Durch diese Methode können wir die Genauigkeit und Robustheit der Vorhersagen verbessern.

\subsubsection{Multiklassifkation}

Notiz: Diskussion zur Loss Funktion und Algorithmus für Multiklassifkation
statt \acrlong{bce} können wir für Multiklassifkationen die \acrlong{ce} berücksichtigen. 
Für Prediction, mit 0.5 Tresholding, kann man bei Multiklassifkationen Softmax nehmen und anschliessend Argmax. 

\subsubsection{Bildmaske}

Schwarze Pixel entfernen, da diese kein Informationen enthalten aber für Perturbationen anfällig sind. 

\subsubsection{Bildanalysen}
Detektion von Adversarial Bilder durch Bildanalysen, wie Fouriertransformation. 

\subsection{Ein Wort zu Technology Credulity}
Experten sollten bei kritischen Use Cases unbedingt in-the-loop bleiben und Ergebnisse kritisch hinterfragen.
