\subsection{Preprocessing}  \label{chap:preprocessing}

Der Preprocessing Schritt erlaubt es uns, die Rohdaten so zu transformieren, dass sie optimal für das Training oder die Verarbeitung durch unser Modell vorbereitet sind. In diesem Kapitel wird insbesondere auf zwei Arten von Preprocessing eingegangen: Preprocessing vor dem Training \ref{chap:Preprocessing vor dem Training} und Preprocessing mit Perturbation \ref{chap:Preprocessing mit Perturbation}.

\subsubsection{Preprocessing vor dem Training} \label{chap:Preprocessing vor dem Training}

Alle Bilder werden vor dem Training durch eine Preprocessing Pipeline vorbereitet. Dabei werden die Bilder, zu einer Grösse von 224 Pixelhöhe und 224 Pixelbreite, mit Antialiasing verarbeitet. 

Antialiasing ist eine Technik, die in der Computergrafik verwendet wird, um den Aliasing-Effekt zu entfernen. Der Aliasing-Effekt ist das Auftreten von gezackten Kanten oder "`Jaggies"' in einem gerasterten Bild.

\subsubsection{Preprocessing mit Perturbation} \label{chap:Preprocessing mit Perturbation}

Die erzeugten Perturbationen werden durch eine elementweise Addition dem jeweiligen Eingangsbild hinzugefügt. Dadurch werden die Bilder manipuliert und für den Angriff auf die Modelle vorbereitet. Die Addition der Perturbationen kann durch das Erstellen eigener PyTorch Custom Preprocessing Klassen codiert werden. Dies erlaubt uns, diese einfach in die Pipeline hinzuzufügen. 



