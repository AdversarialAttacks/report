\section{Methodik}

In diesem Abschnitt widmen wir uns den angewandten Methoden unserer Arbeit und erläutern die verwendeten Modelle sowie ihre Metriken und Algorithmen.

\subsection{Preprocessing}

Alle Bilder werden vor dem Training sowie nach dem Training und Evaluierung, wie folgt vorbereitet. \\

\begin{mdframed}
\begin{lstlisting}[language=python, label={ImagePreprocessor}, caption={Image Preprocessing}]

transform = torchvision.transforms.Compose(
    [
        torchvision.transforms.Resize((224, 224), antialias=True),
    ]
)

\end{lstlisting}
\label{code:preprocessing der Bilder}
\end{mdframed}

Die Bilder werden dabei auf eine Grösse von 224 Pixelhöhe und 224 Pixelbreite mit Antialiasing. 

Antialiasing ist eine Technik, die in der Computergrafik verwendet wird, um den Aliasing-Effekt zu entfernen. Der Aliasing-Effekt ist das Auftreten von gezackten Kanten oder "`Jaggies"' in einem gerasterten Bild.

\subsection{Preprocessing mit Perturbation}
\todo{Ben}