\section{Methodik} 
In diesem Abschnitt werden die in dieser Thesis angewandten Methoden erläutert, einschliesslich der verwendeten Modelle, Metriken und Algorithmen.

\subsection{Technische Umsetzung} \label{chap:technische-umsetzung}
Für die technische Umsetzung dieser Bachelorarbeit wurden bewährte Data Science Frameworks verwendet, darunter Numpy, Pandas, Matplotlib und Seaborn. Für die Modellierung und Bildverarbeitung kamen PyTorch und PyTorch-Lightning zum Einsatz. Der Trainingsprozess sowie die Modelle wurden mit Weights \& Biases verfolgt.

\subsection{Preprocessing}  \label{chap:preprocessing}
Das Preprocessing transformiert die Rohdaten, dass sie optimal für das Training oder die Verarbeitung durch das Modell vorbereitet sind. Dieser Abschnitt behandelt zwei Arten des Preprocessings, Preprocessing beim Training und Preprocessing mit Perturbationen.

\subsubsection{Preprocessing beim Training} \label{chap:Preprocessing beim Training}
Alle Bilder werden beim Training durch eine Preprocessing Pipeline vorbereitet. Dabei werden die Bilder auf eine Grösse von 224 x 224 Pixel mit Antialiasing skaliert.

Antialiasing ist eine Technik, die in der Computergrafik verwendet wird, um den Aliasing-Effekt zu entfernen. Der Aliasing-Effekt führt zu gezackten Kanten in gerasterten Bildern.

\subsubsection{Preprocessing mit Perturbation} \label{chap:Preprocessing mit Perturbation}
Die erzeugten Perturbationen werden durch eine elementweise Addition auf den Inputbildern hinzugefügt. Die Addition der Perturbationen kann durch das Erstellen eigener PyTorch-Custom-Preprocessing-Klassen codiert werden, was eine einfache Integration in die Adversarial Training Pipeline ermöglicht. 
