\section{Einleitung}

Deep Learning ist ein wesentliches Teilgebiet des Maschinellen Lernens und der Künstlichen Intelligenz. In der heutigen Zeit nehmen Machine- und Deep Learning-Modelle eine immer zentralere Stellung in der Gesellschaft ein und werden als Schlüsseltechnologien der Industrie 4.0 angesehen. Die Fähigkeit, aus Daten zu lernen, ermöglicht diesen Modellen, in verschiedenen Anwendungsbereichen wie im Gesundheitswesen, der visuellen Erkennung, der Textanalyse und dem Autonomen Fahren einen erheblichen Mehrwert zu generieren \cite{sarker_deep_2021}. Trotz der faszinierende Möglichkeit und Erfolge die uns die Künstliche Intelligenz bringt, wurde eine beunruhigende Eigenschaft dieser Modelle festgestellt. Forscher haben herausgefunden, dass durch einfaches hinzufügen von bestimmten kleine Störungen, beispielsweise in Form von Veränderungen eines Pixel in einem Bild, die Deep Learning Modelle extrem schlechte Ergebnisse liefern \cite{szegedy_intriguing_2014}. In der Bildklassifikation können selbst geringfügige Veränderungen vorgenommen werden, die für das menschliche Auge kaum wahrnehmbar sind. Diese können dazu führen, dass das Modell eine falsche Klassifikation voraussagt.\cite{perruchoud_24fs_i4ds27_2023}. 

``In einer Welt, wo ML \& AI mehr und mehr Einfluss auf unser Leben insbesondere in sensitiven Bereichen wie Gesundheitswesen und Finanzen nimmt, sind die Implikationen von Adversarial Attacks von enormer Tragweite \cite{perruchoud_24fs_i4ds27_2023}.`` 