\section*{Hilfsmittelverzeichnis}
\addcontentsline{toc}{section}{Hilfsmittelverzeichnis}

\begin{table}[ht!]
    \centering
    \begin{tabular}{l|l|l}
        
        \hline
        \textbf{Hilfsmittel} & \textbf{Verwendung} & \textbf{Betroffene Stellen} \\ 

        % AI Chatbot
        \hline
        GPT3.5/4/4o & Textüberarbeitung und -korrektur. & Ganze Arbeit \\
        Llama3 & Generation von Codeblöcken, welche über- & \\
        DeepL Write & arbeitet und übernommen wurden. & \\

        \hline
        GitHub Copilot & Autovervollständigung von Quellcode & Quellcode \\ 

        % Documentation
        \hline
        Zotero & Quellenverwaltung & Kurzbelege im Fliesstext \\ 
         & & Literaturverzeichnis \\
         
        \hline
        Mermaid & Technische Visualisierungen & Abbildungen (\ref{fig:05-uap_algorithm}, \ref{fig:Evaluierungspipeline})\\

        \hline
        Overleaf & Bericht schreiben & Bericht \\

        \hline
        LanguageTool & Textkorrektur & Bericht \\

        % Communication
        \hline
        Microsoft Teams & Kommunikationskanal & Kommunikation mit Betreuer \\

        % Programming
        \hline
        GitHub & Code Verwaltung und Taskboard & Quellcode und Projekt- \\
        & & management \\

        \hline
        Weights \& Biases & Loggen von Metriken und Modelle & Quellcode \\

        \hline
        Python & Technische Umsetzung der Arbeit &  Quellcode \\
        \& Libraries & Verwendete Libraries: Kapitel \ref{chap:technische-umsetzung} & \\
        
        \hline
        Lambda Labs & Ausführung des Quellcodes & Quellcode \\
        i4DS Slurm & & \\
        Jupyterhub & & \\
        
    \bottomrule
    \end{tabular}
    \caption{Hilfsmittelverzeichnis der Bachelor Thesis}
    \label{tab:hilfmittelverzeichnis}
\end{table}

