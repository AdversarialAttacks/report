\section{Glossar}

Damit die Wörter auftauchen im Glossar oder Akronyme, müssen dabei folgende befehle ausgeführt werden:

\textbackslash Gls\{algorithmus\} \Gls{algorithmus}     \\ 
\textbackslash gls\{algorithmus\} \gls{algorithmus}     \\
\textbackslash Glspl\{algorithmus\} \Glspl{algorithmus} \\

\textbackslash acrlong\{uap\} \acrlong{uap}     \\
\textbackslash acrshort\{uap\} \acrshort{uap}   \\
\textbackslash acrfull\{bce\} \acrfull{bce}     \\

% Wörterglossar

\newglossaryentry{latex}
{
        name=latex,
        description={Is a mark up language specially suited for 
scientific documents}
}

\newglossaryentry{maths}
{
        name=mathematics,
        description={Mathematics is what mathematicians do}
}

\newglossaryentry{formula}
{
        name=formula,
        description={A mathematical expression}
}

\newglossaryentry{algorithmus}
{
        name=algorithmus,
        description={Eine präzise Anweisung oder eine Folge von Anweisungen zur Lösung eines Problems oder zur Ausführung einer Aufgabe in endlicher Zeit}
}

\newacronym{uap}{UAP}{Universal Adversarial Perturbation}

\newacronym{bce}{BCE}{Binary Cross Entropy}

\newacronym{ce}{CE}{Cross Entropy}


%\printglossary[type=\acronymtype]

\printglossaries