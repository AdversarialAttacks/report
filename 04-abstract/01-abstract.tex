\section*{Abstract} 
Adversarial Attacks, insbesondere \acrlong{uap}s (\acrshort{uap}s), stellen eine grosse Herausforderung für die Robustheit von Deep Learning Modellen dar. Diese Thesis konzentriert sich auf die Entwicklung und Implementierung von \acrshort{uap}s und die Evaluierung von Adversarial Training als Verteidigungsmechanismus gegen solche Angriffe, insbesondere für die Modelle ResNet, DenseNet und EfficientNet.

Unser Anwendungsfall zielt auf die Erkennung kritischer Krankheiten durch binäre Klassifikationsmodelle. Im Gegensatz zum Ansatz von Moosavi-Dezfooli et al. \cite{moosavi-dezfooli_universal_2017} lösen wir das Optimierungsproblem der Generierung von \acrshort{uap}s durch eine selbstdefinierte Loss-Funktion, welche die Inverse der \acrlong{bce} und die $L_2$-Matrizennorm verwendet. Bei der Generierung der \acrshort{uap}s werden gezielt positiv gelabelte Bilder angegriffen, um deren Falsch-Negativ-Rate zu erhöhen.

Die Analyse zeigt, dass die Addition von \acrshort{uap}s die Recall-Metriken verschlechtern und dass \Gls{robustifizierung} oft nicht in der Lage ist, die ursprüngliche Modellperformance zu erreichen. Obwohl Adversarial Training die Robustheit erhöht, bleiben die Modelle anfällig für neu generierte \acrshort{uap}s. \Gls{robustifizierung} erzwingt in einigen Fällen grössere Perturbationen, dadurch wird die Zeitintensität der \acrshort{uap}-Generierung erhöht.

Insbesondere im Gesundheitswesen, aber auch in anderen Bereichen, kann eine erhöhte Falsch-Negativ-Rate gefährliche Folgen haben. Daher ist weitere Forschung auf dem Gebiet der Adversarial Attacks und ihrer \Gls{robustifizierung} von entscheidender Bedeutung. Diese Thesis unterstreicht die Notwendigkeit, die Widerstandsfähigkeit gegen Adversarial Attacks zu verbessern.

\clearpage

\section*{Danksagung}
Wir möchten an dieser Stelle all jenen danken, die zur erfolgreichen Fertigstellung unserer Bachelorthesis zum Thema ``Adversarial Attacks - Wie kann KI überlistet werden'' beigetragen haben.

Unser besonderer Dank gilt zunächst Daniel Perruchoud und Stephan Heule von der Fachhochschule Nordwestschweiz, deren fachliche Unterstützung und wertvolle Ratschläge massgeblich zum Gelingen dieser Thesis beigetragen haben. Ihre Expertise und Geduld haben uns in jeder Phase der Thesis entscheidend weitergeholfen.

Auch möchten wir dem Studiengang Data Science und dem Institut für Data Science der Fachhochschule Nordwestschweiz danken. Die breitgefächerte und fundierte Ausbildung sowie die bereitgestellten Ressourcen haben einen wesentlichen Beitrag zur erfolgreichen Durchführung dieser Thesis geleistet.

Ein herzliches Dankeschön geht auch an unsere Freunde, Eltern und Mitstudierenden. Ihre moralische Unterstützung und ihr stetiger Zuspruch haben uns während des gesamten Arbeitsprozesses getragen und ermutigt.




