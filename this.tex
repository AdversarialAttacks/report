\section{Kapitel This}

\lipsum[9]

\subsection{Untertitel 1}

\lipsum[10]

\subsection{Untertitel 2}

\lipsum[11]

\subsubsection{Untertitel 3.~Grades}

Tabelle~\ref{tab:beispieltabelle} zeigt etwas. \lipsum[14-14]

\begin{table}
  \begin{center}
    \renewcommand{\arraystretch}{1.2}
    \begin{tabular}{l|S|S}\hline
      Medium & {Dichte $\rho$ in \SI{}{\kilogram\per\cubic\meter}} & 
               {Schallgeschwindigkeit $c$ in \SI{}{\meter\per\second}}\\
      \hline
		Luft \SI{0}{\degree C} trocken & 1.293 & 331 \\
		Wasserstoff \SI{0}{\degree C} & 0.090 & 1260 \\
		Wasser \SI{0}{\degree C} & 1000 & 1400 \\
		Holz & 600 & 4500 \\
		Stahl & 7700 & 5050 \\
		\hline
    \end{tabular}
  \end{center}
  \caption[Schallgeschwindigkeit in verschiedenen Medien]{Schallgeschwindigkeit in verschiedenen Medien gemäss \cite[S.~566]{hering_physik_2016}}
  \label{tab:beispieltabelle}
\end{table}

\lipsum[15-15]

\subsubsection{Untertitel 3.~Grades}

\lipsum[16-17]